\documentclass[11pt,letterpaper]{article}
\usepackage[margin=.75in]{geometry}
\usepackage{amsmath}
\usepackage{graphicx}
\usepackage{amssymb} \usepackage{natbib}
\usepackage{float} \usepackage{appendix}
\usepackage{hyperref}
\floatstyle{ruled} \restylefloat{table} \restylefloat{figure}
\bibliographystyle{unsrtnat}

\newcommand{\floatintro}[1]{
  
  \vspace*{0.1in}
  
  {\footnotesize

    #1
    
  }
  
  \vspace*{0.1in} }
\newcommand{\Hline}{\noindent\rule{18cm}{0.5pt}} \title{Homework 3:
  Industrial Organisation} \author{Dhananjay Ghei} \date{November 27,
  2018}
\begin{document}
\maketitle
\section*{Question 1}
There are $J_t+1$ goods in each year $t=1971,\dots,1990$ including the
outside good. Assume utility is given by
\begin{align*}
  u_{ij} = \alpha (y_i-p_j) + x_j \beta + \epsilon_{ij} \qquad j=0,\dots,J_t
\end{align*}
with $\epsilon_{ij}$ type 1 extreme value, $y_i$ income, $x_j$, $p_j$
observed characteristics, and parameters $\beta, \alpha$. Suppose we
observe aggregate data - market shares - as opposed to the micro-level
data of zeros and ones (no purchase, purchase).  Derive the
appropriate likelihood function for the aggregated data by starting
with the micro-level specification and regrouping. \\ \Hline \\
\textit{N.B.} The code for this exercise was written in
\texttt{R} and is available on my Github
account. \url{github.com/dhananjayghei/io_estimation}.

\section*{Question 2}
Using the automobile data, estimate the logit demand specification using
maximum likelihood and assuming prices are exogenous. What is the implied
own-price elasticity of the 1990 Honda Accord (HDACCO)? What is the
implied cross-price elasticity of Honda Accord with respect to the 1990 Ford
Escort (FDESCO)? Pick two additional cars and report the same
numbers. \\ \Hline \\



\begin{table}[htbp!]
  \floatintro{The table reports the own-price and cross-price
    elasticities of different products using the maximum likelihood estimation.}
  \centering
    \input{./tables/mle_elast_float.gen}
  \caption{Elasticities from Maximum Likelihood Estimation}
  \label{tab:mle_elas}
\end{table}


\begin{table}[htbp!]
  \floatintro{The table reports the coefficients and standard errors
    from maximum likelihood estimation.}
  \centering
    \input{./tables/mle_est_float.gen}
  \caption{Results with Maximum Likelihood Estimation}
  \label{tab:mle_est}
\end{table}

\section*{Question 3}
Estimate the logit demand specification using the linearised version
of this model from BLP. What is the implied own-price elasticity of
the 1990 Honda Accord (HDACCO)? What is the implied cross-price
elasticity of Honda Accord with respect to the 1990 Ford Escort
(FDESCO)? Pick two additional cars and report the same numbers. \\
\Hline \\

First, I try to replicate the descriptive statistics of Table 1 from
BLP. In particular, for all the variables in the dataset, I calculate
the sales weighted mean and report them. Table \ref{tab:desc_stats}
reports the descriptive statistics and it matches the Table 1 from the
BLP paper.
\begin{table}[htbp!]
  \floatintro{The table reports the descriptive statistics of the data
  similar to Table 1 of BLP. The columns are sales weighted mean of
  the variables in each year.}
  \centering
  \resizebox{.5\textwidth}{!}{
  \input{./tables/blp_table1_float.gen}}
  \caption{Descriptive Statistics}
  \label{tab:desc_stats}
\end{table}

Next, I convert the logit demand specification into a linearised
version by taking logarithms. Following this transformation, we can
run OLS to estimate the logit demand. The results for the regression
are reported in Table \ref{tab:ols_iv}. The results from OLS match
Column 1 from Table III of \citet{blp} (BLP, hereafter). \\
Subsequently, I calculate the elasticities for 4 cars - namely, BMW
735i, Ford Escort, Honda Accord and Volkswagen Jetta in 1990. Table
\ref{tab:ols_elas} shows the own and cross-price elasticities for
these products. As one would expect, the own price elasticities are
negative and of the right sign. The cross-price elasticities are the
same for each car w.r.t the other cars. This is because of the logit
specification. Under the logit specification, the cross-price elasticity of product $j$
w.r.t. product $k$ is given by:
\begin{align*}
  \varepsilon_{jk} = \frac{\partial s_j}{\partial p_k} \frac{p_k}{s_j}
  = \alpha p_k s_k
\end{align*}
\begin{table}[htbp!]
  \floatintro{The table shows the own-price and cross-price
    elasticities of different products using the ordinary least squares
    regression.}
  \centering
  \input{./tables/ols_elast_float.gen}
  \caption{Elasticities from Ordinary Least Squares}
  \label{tab:ols_elas}
\end{table}
\section*{Question 4}
Use the instruments used in Berry-Levinsohn-Pakes. You will need the
firmids and the year variables to calculate these instruments (they
are product firm-year specific). Estimate the logit model using 2SLS
and instrumenting for price. What is the implied own-price elasticity
of the 1990 Honda Accord (HDACCO)? What is the implied cross-price
elasticity of Honda Accord with respect to the 1990 Ford Escort
(FDESCO)? Pick two additional cars and report the same numbers. \\
\Hline \\
One would assert that the price in the OLS model is correlated with
the $\xi_j$ term (unobserved product characteristic) and thus, the OLS
estimates will be biased. Thus, one could think of instrumenting for
price. I follow BLP's approach and instrument for price in the
model. BLP uses three types of instruments:
\begin{enumerate}
\item Observed product characteristics ($z_{rk}$)
\item Sum of observed product characteristics of a single firm in the
  market (excluding the product) ($\sum_{r \neq j} z_{rk}$)
\item Sum of observed product characteristics of rival firms in the
  market ($\sum_{r \neq j, r \neq J_f} z_{rk}$)
\end{enumerate}
I follow this approach and construct the instruments (for all four
product characteristics and the constant term). It is important
to note here that the instruments used in the original BLP paper are
incorrect (as pointed out by Gentzkow and Shapiro in their replication
file) and hence, our results are not directly comparable to the
Column 2 of Table III in BLP. However, we can compare these results
with Column 2 of Table I in \citet{GKP}. Table \ref{tab:ols_iv} shows the
results from the instrumental variable regression. \\
Table \ref{tab:iv_elas} shows the own and cross-price elasticities for
the same 4 cars as mentioned earlier. Note that, instrumenting leads
to an increase in the own-price elasticity when compared with the OLS
estimates.
\begin{table}[htbp!]
  \floatintro{The table shows the own-price and cross-price
    elasticities of different products using the instrumental variable
  regression.}
  \centering
  \input{./tables/iv_elast_float.gen}
  \caption{Elasticities from Instrument Variable Regression}
  \label{tab:iv_elas}
\end{table}
Table \ref{tab:ols_iv} shows the results from the regressions. 
\begin{table}[htbp!]
  \centering
  \resizebox{.7\textwidth}{!}{
  \input{./tables/ols_iv_reg.gen}}
  \caption{Results with Logit Demand}
  \label{tab:ols_iv}
\end{table}

\section*{Question 5}
Write 
\begin{align*}
  u_{ij} = \alpha (y_i-p_j) + x_j \beta_i + \xi_j + \epsilon_{ij}
  \qquad j=0,\dots,J
\end{align*}
Now add random coefficients for each characteristic and estimate the
means and variances of these normally distributed random
coefficients. Estimate the demand side of the model only (unless you
are ambitious and want smaller standard errors - then add the supply
side too). What is the implied own- price elasticity of the 1990 Honda
Accord (HDACCO)?  What is the implied cross-price elasticity of Honda
Accord with respect to the 1990 Ford Escort (FDESCO)? Pick two
additional cars and report the
same numbers. \\ \Hline \\
While the estimates look better with instrumental variable
regressions, the simple model still does not generate the
substitution patterns as seen in the data. In order to correct for
this, we introduce the random coefficients model. 
\begin{table}[htbp!]
  \floatintro{The table shows the own-price and cross-price
    elasticities of different products using the random coefficients model.}
  \centering
  \input{./tables/rand_coef_elast_float.gen}
  \caption{Elasticities from Random Coefficients Model}
  \label{tab:rc_elas}
\end{table}

\section*{Question 6}
Now add income effects as another random coefficient to the model so
the utility specification is given by
\begin{align*}
  u_{ij} = \alpha \log(y_i-p_j) + x_j \beta_i + \xi_j + \epsilon_{ij}
  \qquad j=0,\dots,J
\end{align*}
Draw $y_i$ from a log-normal with a mean that varies by year for
1971-1990 given by (2.01156, 2.06526, 2.07843, 2.05775, 2.02915,
2.05346, 2.06745, 2.09805, 2.10404, 2.07208, 2.06019, 2.06561,
2.07672, 2.10437, 2.12608, 2.16426, 2.18071, 2.18856, 2.21250,
2.18377) and a fixed variance given by $\sigma_y = 1.72$. What is the
implied own-price elasticity of the 1990 Honda Accord (HDACCO)? What
is the implied cross-price elasticity of Honda Accord with respect to
the 1990 Ford Escort (FDESCO)? Pick two additional cars and report the
same numbers. \\ \Hline \\
Table \ref{tab:rc_elas_inc} reports the own and cross-price
elasticities of different cars using the random coefficients model and
the income draws. One can see that the elasticity numbers go up and
the substitution patterns as seen in the earlier logit model
specification change as well. The elasticities start to look more
reasonable as we introduce unobserved heterogeneity into the model. 
\begin{table}[htbp!]
  \floatintro{The table shows the own-price and cross-price
    elasticities of different products using the random coefficients
    model and the income draws.}
  \centering
  \input{./tables/rand_coef_elast_inc_float.gen}
  \caption{Elasticities from Random Coefficients Model}
  \label{tab:rc_elas_inc}
\end{table}

\section*{Question 7}
Put together a table with the point estimates and standard errors for all
of your estimates. In the case of 5 and 6, you do not need to compute
the standard errors for the parameters outside of $\delta$, but do
report the standard errors on the parameters inside of $\delta$, as
you do in the exercise 2-4. \\ \Hline\\
Table \ref{tab:mle_est} reports the maximum likelihood estimation
results. Table \ref{tab:ols_iv} reports the results from the demand
side for the OLS and IV models. Table \ref{tab:estimates} reports the
coefficients estimated from the random coefficients model.
\begin{table}[htbp!]
  \floatintro{The table shows the estimates of the parameters using
    random coefficients model using the BLP specification on 2217
    observations.}
  \centering
  \begin{tabular}{lrrrrr}
    \hline
    Demand Side & Variable & \multicolumn{2}{r}{No Demographics} &
                                                                   \multicolumn{2}{r}{With Demographics} \\
    \hline
                &  & Parameter & Std Error & Parameter & Std Error \\ \hline
    Mean ($\bar \beta$)s & Constant & -9.261 & 1.444 & -10.587 & 1.186 \\
                & HP/Weight & 4.269 & 1.358 & 5.232 & 1.090 \\
                & Air & 0.140 & 0.918 & -1.612 & 3.039 \\
                & MPD & -0.314 & 0.343 & 0.004 & 0.404 \\
                & Size & 0.453 & 0.825 & 0.525 & 0.862 \\
    Std. Deviations ($\sigma_{\beta}$)s & Constant & 3.644 & 0.883 & -1.782 & 0.816 \\
                & HP/Weight & -0.535 & 2.895 & 1.832 & 1.093 \\
                & Air & 1.910 & 1.011 & -3.752 & 2.419 \\
                & MPD & 0.305 & 0.330 & 0.436 & 0.420 \\
                & Size & 2.840 & 0.436 & -2.154 & 0.560 \\
    Term on Price & $\log(y-p)$ &  &  & -0.238 & 0.031 \\ 
    \hline
  \end{tabular}
  \caption{Estimated parameters of the demand equation - Random
    Coefficients Model}
  \label{tab:estimates}
\end{table}

\newpage
\bibliography{IOpapers}

\end{document}
