\documentclass[11pt,letterpaper]{article}
\usepackage[margin=.75in]{geometry}
\usepackage{amsmath}
\usepackage{graphicx}
\usepackage{amssymb} \usepackage{natbib}
\usepackage{float} \usepackage{appendix}
\usepackage{hyperref}
\usepackage{mathrsfs}
\floatstyle{ruled} \restylefloat{table} \restylefloat{figure}
\bibliographystyle{unsrtnat}

\newcommand{\floatintro}[1]{
  
  \vspace*{0.1in}
  
  {\footnotesize

    #1
    
  }
  
  \vspace*{0.1in} }
\newcommand{\Hline}{\noindent\rule{18cm}{0.5pt}} \title{Homework 5:
  Industrial Organisation} \author{Dhananjay Ghei} \date{December 11,
  2018}
\begin{document}
\maketitle
\textit{N.B. The code for this exercise was written in
\texttt{R} and is available on my Github
account. \url{www.github.com/dhananjayghei/io_estimation}.}\\
Problems 1 and 2 use the automobile data from Berry-Levinsohn-Pakes
(1995). This data already accounts for the outside good (sum shares in
any year and they will be less than one.)
\section*{Question 1}
Replicate as closely as you can column 1-3 from GKP. Produce a table
that looks similar and fill in with the results that you find. For
column 3 do not worry about correcting standard errors for first two
stages of estimation (i.e. just use results from non-linear least
squares search). They will differ from results reported in the paper
which are corrected for the first two stages of estimation. \\ \Hline \\
Note that, columns I and II from GKP are the same as the ones in
BLP homework. 
\section*{Question 2}
Reproduce Table 2 using your preferred estimates from column 3 of
Table 1 of your results. \\ \Hline \\
% \newpage
% \bibliography{IOpapers}

\end{document}
