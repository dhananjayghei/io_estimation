\documentclass[11pt,letterpaper]{article}
\usepackage[margin=.75in]{geometry} \usepackage{amsmath}
\usepackage{graphicx} \usepackage{amssymb} \usepackage{natbib}
\usepackage{float} \usepackage{appendix} \usepackage{hyperref}
\usepackage{mathrsfs} \floatstyle{ruled} \restylefloat{table}
\restylefloat{figure} \bibliographystyle{unsrtnat}

\newcommand{\floatintro}[1]{
  
  \vspace*{0.1in}
  
  {\footnotesize

    #1
    
  }
  
  \vspace*{0.1in} } \newcommand{\Hline}{\noindent\rule{18cm}{0.5pt}}
\title{Homework 5: Industrial Organisation} \author{Dhananjay Ghei}
\date{December 11, 2018}
\begin{document}
\maketitle
\textit{N.B. The code for this exercise was written in \texttt{R} and
  is available on my Github
  account. \url{www.github.com/dhananjayghei/io_estimation}.}\\
Problems 1 and 2 use the automobile data from Berry-Levinsohn-Pakes
(1995). This data already accounts for the outside good (sum shares in
any year and they will be less than one.)
\section*{Question 1}
Replicate as closely as you can column 1-3 from GKP. Produce a table
that looks similar and fill in with the results that you find. For
column 3 do not worry about correcting standard errors for first two
stages of estimation (i.e. just use results from non-linear least
squares search). They will differ from results reported in the paper
which are corrected for the first two stages of estimation. \\ \Hline \\

Note that, columns I and II from \citep{GKP}(GKP, hereafter) are the
same as the ones in BLP homework. I construct the instruments in the
same way as in the previous homework. Once again, note that the
instruments are constructed correctly. Table \ref{tab:gkp1} shows the
results of OLS and 2SLS regression on the automobile data set. These
columns are the same as in Table I of GKP.

Next, I construct the $\xi$ variable and use these to construct
$\xi_{(1)}$ and $\xi_{(2)}$ in the same way as in GKP by refining the
$\xi$ similar to the instruments. Following this, I construct the
controls as follows:
\begin{align*}
  V_1 = \xi && V_2 = \xi^2-E[\xi^2 | Z_{j}] && V_3 = \xi^3 -
                                               E[\xi^3 | Z_j]\\
  V_4 = \xi_{(1)} && V_5 = \xi_{(1)}^2 - E[\xi_{(1)}^2 | Z_j]
                                            && V_6 = \xi_{(1)}^3 - E[\xi_{(1)}^3 | Z_j]\\
  V_7 = \xi_{(2)} && V_8 = \xi_{(2)}^2 - E[\xi_{(2)}^2|Z_j] &&
                                                               V_9 = \xi_{(2)}^3 - E[\xi_{(2)}^3 | Z_j]
\end{align*}
Using these controls and the income means\footnote{I used the income
  means as the ones given in the previous homework.}, I run the
regression as in CMRCF using the non-linear least squares
search. Table \ref{tab:gkp2} shows the results from CMRCF
estimation. I estimated the coefficients using non-linear least
squares. The coefficients for the product characteristics are similar
to the ones in column III of GKP whereas the coefficients for the
controls and the interaction terms do not look similar to the ones in
the paper, perhaps because of the income mean term.

\begin{table}[htbp!]
  \floatintro{The table shows the results from OLS and 2SLS estimation
    on BLP data set. These tables replicate the columns I and II of
    Table I from GKP.}
  \centering
  \footnotesize{
    \input{./tables/gkp_ols.gen}}
  \caption{Estimated Parameters for Automobile Demand (OLS, 2SLS)}
  \label{tab:gkp1}
\end{table}

\begin{table}[htbp!]
  \floatintro{The table shows the results from the CMRCF estimation
    on BLP data set. This table replicate the columns III of
    Table I from GKP.}
  \centering
  \footnotesize{
    \input{./tables/gkp_cmrcf.gen}}
  \caption{Estimated Parameters for Automobile Demand (CMRCF)}
  \label{tab:gkp2}
\end{table}

\section*{Question 2}
Reproduce Table 2 using your preferred estimates from column 3 of
Table 1 of your results. \\ \Hline \\
Table \ref{tab:gkp3} shows the results of elasticity from OLS, 2SLS
and CMRCF as in Table II of GKP. In the OLS case, I was able to
replicate the median, mean, standard deviation and the percent of
inelastic demands for OLS from the full data set. In the data set for
the year 1990, I was able to replicate the median, mean, standard
deviation and the percent of inelastic demand as well.  Moving to the
2SLS case, I was able to replicate the median, mean, standard
deviation from the GKP paper for both the full data set and the data
set from 1990s. However, I get that the percent of inelastic demands
in the 2SLS case for the full data is 33.65\% (compared to the 21\%
from GKP Table II) and for the 1990 data is 19.85\% (compared to the
12\% from GKP Table II). Finally, moving to the CMRCF case, I was able
to replicate similar numbers for the median and mean for the full data
set. The other numbers are slightly off but in the same ballpark as
the estimates of GKP (except for the percent of inelastic demands,
which turns out to be pretty low in my case for both the full data set
and the 1990 data set).
\begin{table}
  \centering
  \input{./tables/gkp_table2.gen}
  \caption{Automobile elasticities: OLS, 2SLS, CMRCF (with
    interactions)}
  \label{tab:gkp3}
\end{table}
Table \ref{tab:gkp3} also reports the elasticities for the 1990 Models
as in BLP Table VI. I was able to replicate the correct elasticities
for these cars from the OLS and 2SLS regressions. The elasticities
have the right sign and are of similar magnitude when I compare my
CMRCF estimation with that from the paper.

\bibliography{IOpapers}

\end{document}
