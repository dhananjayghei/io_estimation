\documentclass[11pt,letterpaper]{article}
\usepackage[margin=.75in]{geometry}
\usepackage{amsmath}
\usepackage{graphicx}
\usepackage{amssymb} \usepackage{natbib}
\usepackage{float} \usepackage{appendix}
\usepackage{hyperref}
\usepackage{mathrsfs}
\floatstyle{ruled} \restylefloat{table} \restylefloat{figure}
\bibliographystyle{unsrtnat}

\newcommand{\floatintro}[1]{
  
  \vspace*{0.1in}
  
  {\footnotesize

    #1
    
  }
  
  \vspace*{0.1in} }
\newcommand{\Hline}{\noindent\rule{18cm}{0.5pt}} \title{Homework 6:
  Industrial Organisation} \author{Dhananjay Ghei} \date{December 11,
  2018}
\begin{document}
\maketitle
\textit{N.B. The code for this exercise was written in
\texttt{R} and is available on my Github
account. \url{www.github.com/dhananjayghei/io_estimation}.}

\section*{Some basics}
\begin{enumerate}
\item Read the data into a statistical package and look at summary
statistics to convince yourself that the data was read in
correctly. Try a simple OLS regression of log(QUANTITY) on a constant,
log(PRICE), LAKES, and (twelve of) the seasonal dummy variables. If
you were to view this as an estimate of a demand curve what would the
price elasticity of demand be? Why does this number seem unreasonable?
\\ \Hline \\
\item Try doing the regression instead using instrumental variable
  with the COLLUSION variable as the instrument for PRICE. How does
  the reported price elasticity change. Is the estimate closer to that
  in Porter's paper or that in Ellison's paper and why? How do you
  interpret the coefficient on the LAKES variable? On the seasonal
  dummies? What is the R-squared of the regression and what do you
  make of it? \\ \Hline \\
\item Try the regression with the DM1-DM4 and COLLUSION as instruments
  for price. Do the estimates ``improve'' in any way? \\ \Hline \\
\item Estimate a supply equation as in Porter and Ellison using the
  LAKES variable as an instrument for quantity. What does the
  magnitude of the coefficient on COLLUSION tell us about the effect
  of collusion on prices? What might the coefficient on QUANTITY in
  this regression indicate about the nature of costs in the JEC? \\
  \Hline \\
\end{enumerate}

\section*{Model derivation and interpretation}
\begin{enumerate}
\item Suppose that rather than the log-log specification of demand
  you've been using so far, you tried others and found that a linear
  specification of demand like
  \begin{align*}
    Q_t = \alpha_0 + \alpha_1 P_t + \alpha_2 Lakes_t + u_t
  \end{align*}
seemed most appropriate. Show that for this demand curve the optimal
price for a monopolist with a constant marginal cost of $c$ to set is
\begin{align*}
  P_t = c - \frac{1}{\alpha_1} Q_t
\end{align*}
Given this result, what functional form would you choose for the
supply curve in this model? \\ \Hline \\
\item What pricing rule would result with this demand curve if the
  industry instead consisted of perfectly competitive firms with total
  costs of the form $c(Q_t) = c_0Q_t + c_1Q_t^2$ setting price equal
  to marginal cost? Could one use an approach like Porter's to
  distinguish between these two models of behavior? Talk about why
  this is an important question. \\ \Hline \\
\end{enumerate}
\section*{Causes of price wars}
\begin{enumerate}
\item Using the collusion variable generate an indicator variable for
  the start of a price war. Perform a probit regression with this
  indicator as a dependent variable and with QUANTITY, LAKES, and
  DM1-DM4 (or a subset thereof) as explanatory variables. What
  inferences might you want to draw about whether price wars are more
  likely to occur in booms from the coefficients on the first two
  variables? Why are these variables not really the right ones to be
  using in the equation? \\ \Hline \\
\end{enumerate}
% \newpage
% \bibliography{IOpapers}

\end{document}
