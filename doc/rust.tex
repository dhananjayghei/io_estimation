\documentclass[11pt,letterpaper]{article}
\usepackage[margin=.75in]{geometry} \usepackage{amsmath}
\usepackage{graphicx} \usepackage{amssymb} \usepackage{natbib}
\usepackage{float} \usepackage{appendix} \usepackage{hyperref}
\usepackage{rotating} \usepackage{pdflscape} 
\usepackage{mathrsfs} \floatstyle{ruled} \restylefloat{table}
\restylefloat{figure} \bibliographystyle{unsrtnat}
\setcounter{tocdepth}{1}
\newcommand{\floatintro}[1]{
  
  \vspace*{0.1in}
  
  {\footnotesize

    #1
    
  }
  
  \vspace*{0.1in} } \newcommand{\Hline}{\noindent\rule{18cm}{0.5pt}}
\title{Replicating John Rust's Bus replacement engine paper}
\author{Dhananjay Ghei\footnote{Department of Economics, University of
    Minnesota. This replication paper was written as a part of
    Homework 4 for ECON 8602 - Industrial Organisation. The code for
    this exercise was written in \texttt{R}. Replication files are
    present on GitHub.}}  \date{December 16, 2018}
\begin{document}
\maketitle
\begin{abstract}
  This note replicates the estimation procedure of
  \citet{rust1987optimal} paper in \texttt{R}. We have the data on 8
  groups of buses. This is the same data set as used in the original
  paper from \citet{rust1987optimal}. The paper presents the summary
  statistics to see if the data looks similar to the original paper
  and then proceeds to estimation for different groups of buses. I
  replicate the within and between group estimates for different bus
  groups and then estimate the full likelihood to get estimates for
  three different groups (Group 1,2,3, Group 4, and Group
  1,2,3,4). Instead of using the BHHH algorithm, I use the inbuilt
  optimisation routines of \texttt{R} and the estimates are quite
  close to the ones in the original paper.
\end{abstract}
\newpage
\tableofcontents
\newpage
\section{Data}
The data is available on Rust's
website\footnote{\url{https://editorialexpress.com/jrust/nfxp.html}}. The
data is taken from \citet{rust1987optimal} paper. The data is
available on 9 bus groups. However, in the estimation procedure,
\citet{rust1987optimal} uses only the maintenance records of 162 buses
excluding the bus group ``D309''. I will drop this group as in the
original paper and follow the procedure closely. I
downloaded the zip file and read it in \texttt{R}.\\
The data is available from the period December 1974 until May
1985. The data is provided in a single column format which needs to be
reshaped in the correct dimensions. Following this, the first 11 rows
for each bus in each data set contains the data on the following
items:
\begin{enumerate}
\item Bus Number
\item Month Purchased
\item Year Purchased
\item Month of 1st engine replacement
\item Year of 1st engine replacement
\item Odometer at replacement
\item Month of 2nd engine replacement
\item Year of 2nd engine replacement
\item Odometer at replacement
\item Month (Odometer data begins)
\item Year (Odometer data begins)
\end{enumerate}
Subsequently, the remaining rows are the monthly mileage observations
for each bus. Table \ref{tab:sumstats} gives the summary statistics of
the bus types included in the sample. This is consistent with Table
IIa from \citet{rust1987optimal} therefore, we have the correct data
for estimation. One can see that a lot of these numbers are similar to
the ones in the original paper thereby, guaranteeing that the data was
read in correctly.

\begin{table}
  \floatintro{The table shows summary statistics of for the subsample
    of buses for which at least 1 replacement occurred. The table
    corresponds to the Table IIa in \citep{rust1987optimal}}. 
  \centering
  \input{./tables/stats_rust.gen}
  \caption{Summary of replacement data}
  \label{tab:sumstats}
\end{table}

Finally, the data consists of $\{i_t, x_t\}$ where $i_t^m$ is the
engine replacement decision in month $t$ for bus $m$ and $x_t^m$ is
the mileage since last replacement of bus $m$ in month $t$. In the
next section, the goal is to estimate the parameters
$\theta = (\beta, \theta_1, RC, \theta_3)$ by maximum likelihood using
the nested fixed point algorithm.
\section{Methodology}
The first step is to discretise the state space. In order to do so, I
divide the mileage into equally sized ranges of length 5000 with an
upper bound of 450,000 miles. Thus, $n=90$ in this case. Following
this, the state variable $x$ will take only discrete integer values in
the set $\{1,2,3,\dots,n\}$. Using the discretised state variable $x$,
the decision rule is then reduced to a simple multinomial distribution
on the discretised space. In this case, I follow rust and construct
an indicator for mileage which takes only 3 values $\{0,1,2\}$
corresponding to monthly mileage in the intervals $[0,5000), [5000,
10,000)$ and $[10,000, \infty)$ respectively.  
The estimation strategy as laid out in \citet{rust1987optimal} is in
three stages corresponding to each of the likelihood functions $l^1$,
$l^2$ and $l^f$, where $l^f$ is the full likelihood function and $l^1$
and $l^2$ are ``partial likelihood'' functions given by:
\begin{align*}
  l^1(x_1, \dots, x_T, i_1, \dots, i_T | x_0, i_0, \theta) &=
                                                             \Pi_{t=1}^T p(x_t|x_{t-1},i_{t-1},\theta_3)\\
  l^2(x_1, \dots, x_T,i_1,\dots,i_T|\theta) &= \Pi_{t=1}^T P(i_t|x_t, \theta)
\end{align*}
The first step is to estimate $p(x)$ non-parametrically. I do this for
within group and between group estimates as in the original
paper. Given the discretised space, the probability $\pi_j$ is then
given by: $\pi_j = Pr\{x_{t+j}=x_t+j\}$ for $j=0,1,2$. The
non-parametric estimation then involves simply calculating the average
number of times the bus was in a particular state. I calculate these
averages for within group (i.e. for each bus group separately) and for
between group (i.e. for combined bus groups). Standard errors are
calculated based on the standard non-parametric estimation.\\
Once, we estimate these probabilities, we can construct a $n \times n$
transition matrix  $\Pi$ which is given by:
\begin{align*}
  \Pi = \begin{bmatrix}
    \pi_0 & \pi_1 & \pi_2 & 0 & 0 & \dots & 0\\
    0 & \pi_0 & \pi_1 & \pi_2 & 0 & \dots & 0\\
    0 & 0 & \pi_0 & \pi_1 & \pi_2 & \dots & 0\\
    \vdots \\
    0& 0& &\dots & & \pi_0 & 1-\pi_1&\\
    0 & 0 & \dots & 0 & \dots & 0& 1
  \end{bmatrix}
\end{align*}
This is the transition probability matrix for the uncontrolled Markov
process. Now, we can proceed to the next step of estimation.  Finally,
I estimate the parameters $\theta_1$ and $RC$ by MLE using the nested
fixed point algorithm. Set $\beta=.9999$. Instead of using the BHHH
algorithm, I will use the inbuilt optimisation routines in \texttt{R}.
\section{Results}
Table \ref{tab:within} shows the within group estimates of $\theta_3$
form 8 different bus groups. These are estimated non-parametrically by
directly calculating the mean.

\begin{table}[htbp!]
  \floatintro{The table shows the within group estimates of $\theta_3$
    from 8 different groups as in \citet{rust1987optimal}. Standard
    errors are in parantheses. These estimates match the ones in the
    original paper.}
  \centering
  \resizebox{.9\textwidth}{!}{
    \input{./tables/within_est.gen}}
  \caption{Within group estimates}
  \label{tab:within}
\end{table}

Table \ref{tab:between} shows the between group estimates of
$\theta_3$ from the same groups as taken in the paper. The estimates
of probabilities match the ones in the paper. Consider, for example,
the bus group 1,2,3,4 which is what we will be estimating later
on. $\theta_{31}$ equals .349, $\theta_{32}$ equals .572 and
$\theta_{33}$ equals .012.
\begin{table}[htbp!]
  \floatintro{The table shows the between group estimates of $\theta_3$
    from 8 different groups as in \citet{rust1987optimal}. Standard
    errors are in parantheses. These estimates match the ones in the
    original paper.}
  \centering
  \resizebox{\textwidth}{!}{
    \input{./tables/between_est.gen}}
  \caption{Between group estimates}
  \label{tab:between}
\end{table}

Table \ref{tab:estimates} shows the structural estimates for the full
likelihood estimation where the cost function is assumed to be of
linear form $c(x,\theta_1)=.001\theta_{11}x$ and $\beta$ is set to
.9999. These are the same specifications as used in the paper to get
Table IX. Estimates for three different samples are presented as in
the original paper. These samples are Group 1,2,3, Group 4, and Group
1,2,3,4. Consider for example the case of Group 1,2,3,4, my estimates
are $RC=9.792$ and $\theta_{11}=2.6695$ which are close to the
original estimates of $RC=9.7558$ and $\theta_{11}=2.6275$ from Table
IX of \citet{rust1987optimal}. The standard errors are calculated
using the square root of the inverse of the Hessian matrix. The
Hessian from the optimisation routine is the second-derivative of the
likelihood function evaluated at the optimum. The standard errors
estimated for the same group are $se(RC)=.909$ and
$se(\theta_{11})=.4781$ which are similar to the estimates from the
Rust's paper of $se(RC)=1.227$ and $se(\theta_{11})=.618$. The
difference in the standard errors comes from the fact that I am using
the Fischer information matrix (Hessian) instead of the original
method from the Rust paper.
\begin{table}[htbp!]
  \floatintro{The table shows the structural estimates for cost
    function $c(x,\theta_1)=.001\theta_{11}x$. The fixed point
    dimension is 90 and $\beta=.9999$. Standard errors are in parantheses. This table
    replicates Table IX of \citep{rust1987optimal}.}
  \centering
  \input{./tables/estimates_rust.gen}
  \caption{Structural Estimates for linear cost function}
  \label{tab:estimates}
\end{table}
\section{Conclusion}
In this note, I used the bus replacement data of Harlod Zurcher (which
was used by Rust in his paper) to replicate the empirical model as
estimated by John Rust. I feed in the data, get summary statistics to
see if the data was indeed consistent with the paper. Then, I used the
three step estimation as laid out in the paper to replicate the
results as closely as possible. First, I estimate the within and
between group estimates of the transition and then use it to estimate
the full likelihood function assuming a linear cost function as in the
original paper. I used the nested fixed point algorithm to estimate
the results and follow as closely as possible to Rust's approach. The
only difference being that I used the inbuilt optimisation algorithm
instead of writing the BHHH algorithm for estimation.\\
Following this, I was able to replicate the results from Table IX of
the paper. The point estimates are quite close to the original
estimates.
\section{Replication Files}
The replication files are present on my Github
repository. \url{https://www.github.com/dhananjayghei/io_estimation}.
The code was written in \texttt{R}. The data set is present in
\texttt{data/rust-data.zip}. The code is present in the \texttt{src/}
folder. The \texttt{src/} folder has two files:
\begin{enumerate}
\item \texttt{rust\_functions.R} - This file contains all the functions
  that I wrote for estimation which are called in the main file.
\item \texttt{rust.R} - This is the main file that reads in the data
  set from the zip file, performs estimation, and generates the \LaTeX
  table for results shown in the Results section.
\end{enumerate}
Apart from the base packages that already come with \texttt{R}, you
will need the following \texttt{R} packages to run the code: gtools,
xtable

For people on Linux/Mac, there is also a \texttt{Makefile} present
in the repository that you could use to run the code
directly. Clone the Github repository to your local machine, open the
terminal and go to the \texttt{src/} directory and run the following
command: \texttt{make rust}.
\newpage
\bibliography{IOpapers}
\end{document}