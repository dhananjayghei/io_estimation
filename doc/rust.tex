\documentclass[11pt,letterpaper]{article}
\usepackage[margin=.75in]{geometry} \usepackage{amsmath}
\usepackage{graphicx} \usepackage{amssymb} \usepackage{natbib}
\usepackage{float} \usepackage{appendix} \usepackage{hyperref}
\usepackage{rotating} \usepackage{pdflscape}
\usepackage{mathrsfs} \floatstyle{ruled} \restylefloat{table}
\restylefloat{figure} \bibliographystyle{unsrtnat}

\newcommand{\floatintro}[1]{
  
  \vspace*{0.1in}
  
  {\footnotesize

    #1
    
  }
  
  \vspace*{0.1in} } \newcommand{\Hline}{\noindent\rule{18cm}{0.5pt}}
\title{Replicating John Rust's Bus replacement engine paper}
\author{Dhananjay Ghei\footnote{Department of Economics, University of
    Minnesota. This replication paper was written as a part of
    Homework 4 for ECON 8602 - Industrial Organisation. The code for
    this exercise was written in \texttt{R}. Replication files are
    present on GitHub.}}  \date{December 16, 2018}
\begin{document}
\maketitle
\begin{abstract}
This paper replicates the estimation procedure of
\citet{rust1987optimal} paper in \texttt{R}. We have the data for 8
groups of buses starting from . This is the same data set as used in
the original paper from \citet{rust1987optimal}. The paper presents
the summary statistics to see if the data looks similar to the
original paper and then proceeds to estimation for different groups of
buses.
\end{abstract}
\newpage
\tableofcontents
\section{Introduction}

\section{Data}
The data is taken from \citet{rust1987optimal} paper. The data is
available on 9 bus groups. However, in the estimation procedure,
\citet{rust1987optimal} uses only the maintenance records of 162 buses
excluding the bus group ``D309''. The data is available on Rust's
website\footnote{\url{https://editorialexpress.com/jrust/nfxp.html}}. I
downloaded the zip file and read it in \texttt{R}. The data is
available from the period December 1974 until May 1985. The data is
provided in a single column format which needs to be reshaped in the
correct dimensions. Following this, the first 11 rows for each bus in
each data set contains the data on the following items:
\begin{enumerate}
\item Bus Number
\item Month Purchased
\item Year Purchased
\item Month of 1st engine replacement
\item Year of 1st engine replacement
\item Odometer at replacement
\item Month of 2nd engine replacement
\item Year of 2nd engine replacement
\item Odometer at replacement
\item Month (Odometer data begins)
\item Year (Odometer data begins)
\end{enumerate}
Following this, the remaining rows are the monthly mileage
observations for each bus. Table \ref{tab:sumstats} gives the summary
statistics of the bus types included in the sample. This is consistent
with Table IIa from \citet{rust1987optimal} therefore, we have the
correct data for estimation. One can see that a lot of these numbers
are similar to the ones in the original paper thereby, guaranteeing
that the data was read in correctly.

\begin{table}
  \floatintro{The table shows summary statistics of for the subsample
    of buses for which at least 1 replacement occurred. The table
    corresponds to the Table IIa in \citep{rust1987optimal}}. 
  \centering
  \input{./tables/stats_rust.gen}
  \caption{Summary of replacement data}
  \label{tab:sumstats}
\end{table}

Finally, the data consists of $\{i_t, x_t\}$ where $i_t^m$ is the
engine replacement decision in month $t$ for bus $m$ and $x_t^m$ is
the mileage since last replacement of bus $m$ in month $t$. In the
next section, the goal is to estimate the parameters $\theta = (\beta,
\theta_1, RC, \theta_3)$ by maximum likelihood using the nested fixed
point algorithm. 
\section{Methodology}
The estimation strategy as laid out in \citet{rust1987optimal} is in
three stages corresponding to each of the likelihood functions $l^1$,
$l^2$ and $l^f$, where $l^f$ is the full likelihood function and $l^1$
and $l^2$ are ``partial likelihood'' functions given by:
\begin{align*}
  l^1(x_1, \dots, x_T, i_1, \dots, i_T | x_0, i_0, \theta) &=
  \Pi_{t=1}^T p(x_t|x_{t-1},i_{t-1},\theta_3)\\
  l^2(x_1, \dots, x_T,i_1,\dots,i_T|\theta) &= \Pi_{t=1}^T P(i_t|x_t, \theta)
\end{align*}
The first step is to estimate $p(x)$
non-parametrically. I do this for within group and between group
estimates as in the original paper. 

Finally, I estimate the parameters $\theta_1$ and $RC$ by MLE using
the nested fixed point algorithm. Set $\beta=.9999$. Instead of using
the BHHH algorithm, I will use the inbuilt optimisation routines in
\texttt{R}. 
\section{Results}
Table \ref{tab:within} shows the within group estimates of $\theta_3$
form 8 different bus groups. These are estimated non-parametrically by
directly calculating the mean.

\begin{table}[htbp!]
  \floatintro{The table shows the within group estimates of $\theta_3$
    from 8 different groups as in \citet{rust1987optimal}. Standard
    errors are in parantheses. These estimates match the ones in the
    original paper.}
  \centering
  \resizebox{.9\textwidth}{!}{
    \input{./tables/within_est.gen}}
  \caption{Within group estimates}
  \label{tab:within}
\end{table}

Table \ref{tab:between} shows the between group estimates of
$\theta_3$ from the same groups as taken in the paper. The estimates
of probabilities match the ones in the paper. Consider, for example,
the bus group 1,2,3,4 which is what we will be estimating later
on. $\theta_{31}$ equals .349, $\theta_{32}$ equals .572 and
$\theta_{33}$ equals .012. 
\begin{table}[htbp!]
  \floatintro{The table shows the between group estimates of $\theta_3$
    from 8 different groups as in \citet{rust1987optimal}. Standard
    errors are in parantheses. These estimates match the ones in the
    original paper.}
  \centering
  \resizebox{\textwidth}{!}{
    \input{./tables/between_est.gen}}
  \caption{Between group estimates}
  \label{tab:between}
\end{table}

\section{Conclusion}

\newpage
\bibliography{IOpapers}
\end{document}