\documentclass[11pt, letterpaper]{article} \usepackage{amsmath}
\usepackage{amssymb} \usepackage{graphicx} \usepackage{bbm}
\usepackage{geometry} \usepackage{natbib} \usepackage{float}
\usepackage[toc, page]{appendix} \usepackage{hyperref} \usepackage{rotating}
\usepackage{mathrsfs} \floatstyle{ruled} \restylefloat{table}
\restylefloat{figure} \bibliographystyle{unsrtnat}
\newcommand{\floatintro}[1]{
  
  \vspace*{0.1in}
  
  {\footnotesize

    #1
    
  }
  
  \vspace*{0.1in} } \newcommand{\Hline}{\noindent\rule{18cm}{0.5pt}}
\title{How much does it cost to export a movie abroad?}
\author{Dhananjay Ghei\footnote{Department of Economics, University of
    Minnesota. This paper was written as a course requirement for ECON
    8602 - Industrial Organisation. The replication files are present
    on Github. The code for this exercise was written in \texttt{R}
    and \texttt{Julia}. Appendix \ref{appendix:a} lays out the files
    you will need in order to replicate the results from this paper.}}
\date{December 20, 2018}

\begin{document}
\maketitle
\begin{abstract}
  Exporting decision of movie depends on the expected profits that a
  movie will earn abroad. Such estimates are known to the exporter but
  not to the econometrician. Given the predicted revenues, one can
  estimate the fixed costs of exporting which will help in inferring
  if the movie is exported or not. This paper tries to estimate the
  fixed costs of exporting movies using a novel data set of 5500
  movies from 2001-2009. We provide bound estimates on the fixed cost
  using revealed preference moment inequalities as in
  \citet{pakes2015}. We estimate the fixed costs to be varying across
  destination and origin countries. The fixed costs seem to be in the
  range of 2-50 million depending on the destination country. 90\%
  confidence intervals are estimated on the bounds using the
  methodology laid out in \citet{pakes2015}.
\end{abstract}
\newpage
\tableofcontents
\newpage
\section{Introduction}
What is the fixed cost of exporting a movie to a foreign country? This
paper tries to answer this question using a moment inequality approach
by estimating the bounds on the fixed costs. In particular, we use a
revealed preference approach as in \citep{pakes2015}. We use the data
on exports of movies from 5 different countries (France, Germany,
Spain, UK and US) to each other between 2001-2009.  Exporting
decisions of a firm depends on the expected profit it will earn when
serving a foreign market. Such estimates are rarely observed in the
data. Thus, while the firm observes the fixed cost of exporting, the
econometrician does not. Moment inequality approach relies on the
information set available to the firms which are making the exporting
decisions to put a bound on these estimates.

The estimation methodology in the paper is closely related to a
growing body of literature\footnote{\citep{tamer2010} provides a good
  summary of the literature in this field. This paper was particularly
  very helpful for me to understand the workhorse of moment
  inequalities.} in the field of moment inequalities. Some of these
papers include \citet{chernozhukov2013}, \citet{manski2004},
\citet{pakes2010}, \citet{andrews2010}, \citet{holmes2011},
\citet{ho2009} and \citet{sweeting2013}. One benefit of using this
approach is that we need not require the econometrician to have full
knowledge of the exporter's information set. The econometrician can
rely on certain key variables from the information set and yet derive
consistent estimates (or, bounds) based on this smaller information
set. Therefore, in this case the estimates are only partially
identified instead of being point identified. This gives us a more
reasonable range of numbers which can be shown to be consistent in
large samples.

We estimate the fixed costs using three different set of instruments
in the firms' information set - predicted revenues, lagged box office
price in the exporter, and the lagged average box office revenue of
the movies from the exporting country.

The rest of the paper is organised as follows. In Section 2, we
describe the model that is used for constructing the decisions to
export. Section 3 describes the data, some issues at hand while
cleaning the data and the summary statistics on the data. Section 4
lays out the construction for moment inequalities and estimation
procedure for the same. Section 5 discusses the results and Section 6
concludes.
\section{Model}
We follow closely to \citep{dickstein2018} in the model's
notation. Consider a movies' export decision. All movies' of country
$h$ choose whether to export in each of the export markets $j$. In our
case, $h,j=1,2,3,4,5$ where the numbers denote the following
countries, namely - France(1), Germany(2), Spain(3), UK(4), US(5).\\
The firms' decide whether to export the movie to any of the $j$
countries or not. If it decides to export, it pays a fixed exporting
cost. \\
We model the export profits that movie $i$ from country $h$ would
obtain in country $j$ at time $t$ as:
\begin{align*}
  \pi_{ijt} = \eta^{-1}r_{ijt} - f_{ijt}
\end{align*}
where, $\eta$ is the demand elasticity assumed to be constant across
markets, $r_{ijt}$ is the revenue that the movie $i$ earns in country
$j$ at time $t$ and $f_{ijt}$ is the fixed cost of exporting the movie
$i$ to country $j$ in time $t$. \\
We model the fixed export cost in three different ways - firstly, the
fixed costs are assumed to be destination/origin/time invariant;
secondly, the fixed costs are assumed to be destination-country
specific; and thirdly, the fixed costs are allowed to be
destination-origin country specific. These cases are shown below:
\begin{align*}
  f_{ijt} = \begin{cases}
    \beta_0 + \nu_{ijt} & \text{common fixed cost}\\
    \beta_{0j} + \nu_{ijt} & \text{fixed cost (varying by destination)}\\
    \beta_{0ij} + \nu_{ijt} & \text{fixed cost varying by destination
      and origin}
  \end{cases}
\end{align*}
where, $\nu_{ijt}$ is the residual term that the econometrician does
not observe. We assume that the firms know the fixed costs $f_{ijt}$.
We will assume that the error term in the fixed cost $\nu_{ijt}$
follows a normal distribution and is independent of export
determinants i.e.
\begin{align*}
  \nu_{ijt} \perp \mathcal{I}_{ijt} \qquad \nu_{ijt} \sim N(0, \sigma_v^2)
\end{align*}
where, $\mathcal{I}_{ijt}$ is the information set of firm $i$ on
country $j$ at time $t$.\\
The risk-neutral firm $i$ will decide to export to country $j$ in
period $t$ if the expected profit is greater than or equal to zero
i.e. $\mathbb{E}[\pi_{ijt} | \mathcal{I}_{ijt}, \nu_{ijt}] \geq 0$.
The information set $\mathcal{I}_{ijt}$ contains the variables that a
movie $i$ uses to predict the revenue in the country $j$ at time
$t$. These variables could include predicted revenue, prices, etc.\\
The expected profits can be written more explicitly as:
\begin{align*}
  \mathbb{E}[\pi_{ijt}|\mathcal{I}_{ijt}, \nu_{ijt}] = \eta^{-1}
  \mathbb{E}[r_{ijt} | \mathcal{I}_{ijt}, \nu_{ijt}] - \beta_0 - \nu_{ijt}
\end{align*}
Let
$d_{ijt} = \mathbbm{1}\{\eta^{-1}\mathbb{E}[r_{ijt} |
\mathcal{I}_{ijt}, \nu_{ijt}] - \beta_0 - \nu_{ijt} \geq 0\}$
denote the decision of movie $i$ to export to country $j$ at time $t$
where, $\mathbbm{1}$ denotes the indicator function. We can re-write
$d_{ijt}$ as:
\begin{align*}
  d_{ijt} = \mathbbm{1}\{\eta^{-1} \mathbb{E}[r_{ijt} |
  \mathcal{I}_{ijt}] - f_{ijt} \geq 0\}
\end{align*}
where, $r_{ijt}$ is export revenue conditional on entry and $f_{ijt}$
is fixed export cost as before. Denote the agent's expectational error
as
$\varepsilon_{ijt} = r_{ijt} - \mathbb{E}[r_{ijt} |
\mathcal{I}_{ijt}]$ and we will assume that:
\begin{align*}
  \mathbb{E}[\varepsilon_{ijt} | \mathcal{I}_{ijt}] = 0
\end{align*}
Among all the variables in the information set, only a subset of them
will be used. We consider three possible cases - firstly, firm has
perfect foresight and knows the exact predicted revenues from
exporting the movie to a particular country, secondly, the firm has
information on the predicted revenue and the lagged box office price
in the other country and finally, the firm has knowledge on the
predicted revenue, lagged box office price in the other country and
the lagged average box office revenue in the country. The steps for
estimation using this method are laid out in Section 4.
\section{Data}
We have data on movies from 5 different countries - France, Germany,
Spain, UK and US from 2001-2009. There are a total of 5500
observations over the 8 years. The data set contains observations on
the actual revenue in the country where the movie was released and the
predicted revenue if the movie was not exported to the country. We
also have a data set on the average box office ticket price in the 5
countries.
\subsection{Data cleaning}
There are certain discrepancies in the data set which one needs to
take care of before proceeding to estimation. For example, consider
the movie ``Harry Potter and the Sorcerer's Stone'' which is a movie
from UK but has missing value of actual revenues in the UK. This is
because the movie was released in the UK under the name ``Harry Potter
and the Philosopher's Stone'' and therefore, these values are
missing. In particular, we have 643 observations of this sort. We drop
these observations from our analysis as this will bias our
estimates. Thus, we have a total of 4857 observations after
dropping these observations. \\
In addition, there are some other movie titles which are
duplicated. Consider for example, ``Berlin Is in Germany'' which is
labelled both as a German and French movie. There are 19 such
duplicates. This is particularly because the movies were jointly
produced in the two countries. However, estimating the fixed costs
using such observations will lead to double counting in the first two
cases and therefore, we need to drop these observations for the
initial analysis.
\subsection{Summary statistics}
Table \ref{tab:1} shows the yearly average revenues of movies released
in different countries. Revenues are reported in millions of USD. The
revenues have an increasing trend over the years. United States stands
out as the country that has the highest average revenue over the years
amongst all 5 countries.
\begin{table}[htbp!]
  \floatintro{The table shows the average revenues of movies released
    in different countries. The first column shows the year and the
    horizontal row shows the average revenues collected across
    different countries. The values are in millions of US dollars.}
  \centering
  \input{./tables/final_tab2_float.gen}
  \caption{Actual yearly average revenues collected (in millions USD)}
  \label{tab:1}
\end{table}

Table \ref{tab:2} shows the average revenues generated by movies of
one country to the other country. Note that, once again, US movies
earn the most in all 5 countries when compared to the other countries.
\begin{table}[htbp!]
  \floatintro{The table shows the average revenues of movies released
    in different countries. The first column shows the country where
    the movie is from and the horizontal row shows the average
    revenues collected in that country. The values are in millions of
    US dollars.}
  \centering
  \input{./tables/final_tab1_float.gen}
  \caption{Actual cross-country (average) revenues collected (in millions USD)}
  \label{tab:2}
\end{table}
Table \ref{tab:tab3} shows the percentage of movies released in
different countries. Note that, based on our data cleaning the
diagonal elements are 100\%. On an average, 75\% of the US movies are
exported to the other 4 countries. Spain is the highest exporter of
French and German movies.
\begin{table}[htbp!]
  \floatintro{The table shows the percentage of movies released in
    different countries. For example, consider the first row, 100\% of
    the movies from France were released in France, 8.48\% of them
    were released in Germany, 10.95\% were released in Spain, 7.70\%
    of them were released in UK and .96\% of them were released in the
    US.}
  \centering
  \input{./tables/final_tab3_float.gen}
  \caption{Percentage of movies exported to other countries}
  \label{tab:tab3}
\end{table}
Since we also have data on the predicted revenue of movie released in
different countries which are not exported to the other country, we
also report the average predicted revenues in millions of USD. Table
\ref{tab:tab4} shows the predicted average revenues of different
movies in the exporting country.
\begin{table}[htbp!]
  \floatintro{The table shows the predicted revenues of movie released
    in different countries which were not exported to the other
    country. The first column shows the country where the movie is from
    and the horizontal row shows the average predicted revenues that
    could potentially be collected in that country. The values are in
    millions of USD.}
  \centering
  \input{./tables/final_tab4_float.gen}
  \caption{Predicted cross-country (average) revenues (in millions USD)}
  \label{tab:tab4}
\end{table}

\section{Methodology}
I use the revealed preference moment inequalities for estimation of
the lower and upper bounds. The literature used either odds-based or
moments-based inequalities or a combination of both to estimate the
bounds. However, I will do the revealed preferences approach as done
in \citep{pakes2015} as this was the one we were expected to follow in
this paper. For any $\mathcal{Z}_{ijt} \subsetneq \mathcal{I}_{ijt}$,
we define a conditional revealed preference moment inequality as:
\begin{align*}
  \mathcal{M}(\mathcal{Z}_{ijt}, \theta) = \mathbb{E}\bigg[ m(d_{ijt},
  r_{ijt}; \theta) \bigg| \mathcal{Z}_{ijt} \bigg] \geq 0
\end{align*}
where, $\theta$ is the list of parameters to be estimated
$\theta = \{\beta_0, \sigma_{v}\}$ and the moment function $m$ is a
$2 \times 1$ vector of moment conditions defined as:
\begin{align*}
  m_l(.) = -(1-d_{ijt})(\eta^{-1}r_{ijt} - \beta_0) + d_{ijt} \sigma
  \frac{\phi(\sigma^{-1}(\eta^{-1}r_{ijt} -
  \beta_0))}{\Phi(\sigma^{-1}(\eta^{-1}r_{ijt} -\beta_0))}\\
  m_u(.) = d_{ijt}(\eta^{-1}r_{ijt} - \beta_0) + (1-d_{ijt}) \sigma
  \frac{\phi(\sigma^{-1}(\eta^{-1}r_{ijt} -
  \beta_0))}{1-\Phi(\sigma^{-1}(\eta^{-1}r_{ijt} -\beta_0))}
\end{align*}
Note the negative sign in front of the lower bound inequality
($m_l(.)$) which is written in a way that it converts it into an
inequality to be greater than or equal to zero. As we are only
interested in estimating the fixed costs, we calibrate the two
parameter values $\eta$ and $\sigma$ to be equal to 1 and 1
respectively. Following this, we solve the revealed preference
inequalities to estimate the parameter vector $\theta = \beta_0$
consistently. The consistency of the estimate is proven in
\citet{dickstein2018} for any instrument
$\mathcal{Z}_{ijt} \subsetneq \mathcal{I}_{ijt}$ for all $i,j$ and
$t$.\\
Using these moment inequalities, we estimate the bounds using three
sets of instruments:
\begin{enumerate}
\item \textit{Case 1:} The expected revenue share (if the movie was
  released in the country) $(\mathcal{Z}_{ijt} = \{\hat s_{ijt}\})$
\item \textit{Case 2:} The expected revenue share and the lagged box
  office ticket price in the country
  $(\mathcal{Z}_{ijt} = \{ \hat s_{ijt}, p_{jt-1}\})$
\item \textit{Case 3:} The expected revenue share, the lagged box
  office ticket price and the lagged average revenue of the exporting
  country in the other country
  $(\mathcal{Z}_{ijt} = \{\hat s_{ijt}, p_{jt-1}, \bar s_{ijt-1}\})$
\end{enumerate}
We estimate the bounds for the three cases as above. Next, I vary the
fixed costs by destination and destination-origin and estimate the
bounds. Finally, I split the data sample into two different time
periods and estimate the fixed costs based on the
destination. \\
The next step is to get the bounds on the confidence interval. I
follow the same approach as in \citet{dickstein2018} to estimate the
confidence intervals.
\section{Results}
We estimate the fixed costs assuming that the firm has three different
sets of instruments in the information set as explained above. In the
baseline case, I will assume that the fixed costs do not depend on
destination/origin/time. Table \ref{res1} shows the results of the
estimation for the first part. Column I shows the estimates from the
first case where the firm has only the information on the predicted
revenues in each of the country, Column II shows the estimates when
the firm has predicted revenues as well as the lagged prices in the
information set and Column III shows the estimates when in addition to
the above two, the firm also has the lagged average revenue in the
information set.  I also do sensitivity analysis of the results by
varying the number of draws used for estimation of the confidence
interval. The results do not change much as I vary the number of draws
from 10000, 1 million to 10 million. The bounds do tend to get a bit
lower as we increase the number of draws. Therefore, in the subsequent
analysis I focus on the first case (i.e. predicted revenues as
instruments) with 10 million draws.

\begin{table}[htbp!]
  \floatintro{The table shows the bound estimates of common fixed
    costs. 90\% confidence intervals are estimated below the
    bounds. Sensitivity analysis is done by varying the number of
    draws to get the confidence interval. All values are reported in
    millions of USD.}
  \centering
  \begin{tabular}{lrrr}
    \hline
    & Case 1 & Case 2 & Case 3 \\
    \hline
    $\beta_0$ & {[}3.002, 9.647{]} & {[}2.801, 9.288{]} & {[}6.247, 9.288{]} \\
    90\% CI &  &  &  \\
    R = 10 million & {[}-16.679, 71.067{]} & {[}-15.649, 66.807{]} & {[}-49.783, 66.457{]} \\
    R = 1 million & {[}-16.527, 71.152{]} & {[}-15.647, 66.794{]} & {[}-49.979, 66.794{]} \\
    R = 10000 & {[}-16.7, 70.637{]} & {[}-15.505, 66.887{]} &
                                                              {[}-50.903, 66.403{]}\\
    \hline
  \end{tabular}
  \caption{Bound estimates of common fixed costs (million USD)}
  \label{tab:res1}
\end{table}
One would posit that the fixed export costs can vary by
destination. Therefore, I estimate the fixed costs varying by the
destination. Table \ref{tab:res2} shows the bound estimates for fixed
costs varying by destination. The results improve slightly as one can
see that there is more variation in the fixed costs compared to the
previous case. The fixed costs of exporting a movie to France, Germany
and Spain is cheaper compared to when the movie is exported to the UK
or US. Splitting by the destination country shows that the costs in
the previous case did not highlight this variation.
\begin{table}[htbp!]
  \floatintro{The table shows the bound estimates of fixed costs
    varying by destination using only the first set of instruments
    ($\mathcal{Z}_{ijt} = \hat s_{ijt}$). 90\% confidence intervals are reported
    below the bounds. 10 million draws were used in the estimation of
    confidence intervals. All values are
    reported in millions of USD.}
  \centering
  \resizebox{\textwidth}{!}{
    \begin{tabular}{lrrrrr}
      \hline
      & France & Germany & Spain & UK & US \\
      \hline
      $\beta_{0j}$ & {[}0.283, 2.89{]} & {[}0.206, 3.817{]} & {[}0.21, 2.465{]} & {[}0.263, 5.255{]} & {[}11.313, 53.151{]} \\
      90 \% CI & {[}-1.61, 15.221{]} & {[}-1.194, 21.751{]} & {[}-0.975, 11.479{]} & {[}-2.825, 29.471{]} & {[}-23.562, 192.553{]}\\
      \hline
    \end{tabular}}
  \caption{Bound estimates of fixed costs varying by destination
    (million USD)}
  \label{tab:res2}
\end{table}
One could also argue that the fixed costs vary not only by the
destination but also by the origin country. Subsequently, I estimate
the fixed costs varying by both the destination and origin. Table
\ref{tab:res3} shows the bound estimates for this exercise. Note that
the lower bound of confidence intervals in the diagonal will be ``NA''
as in the diagonal the origin and the destination country is the same
and the way we cleaned the data set earlier, there are no movies of
the origin country in the data set which are not released in the same
country itself.
\begin{table}[htbp!]
  \floatintro{The table shows the bound estimates of fixed costs
    varying by both destination and origin using only the first set of
    instruments. 90\% confidence intervals are reported below the
    bounds. 10 million draws were used in the estimation of confidence
    intervals. All values are reported in millions of USD. The rows correspond to the
    origin and the columns correspond to the destination.}
  \centering
  \resizebox{\textwidth}{!}{
    \begin{tabular}{lrrrrr}
      \hline
      & France & Germany & Spain & UK & US \\
      \hline
      France & {[}0.0, 1.861{]} & {[}0.169, 1.53{]} & {[}0.083, 0.894{]} & {[}0.1, 1.295{]} & {[}12.108, 31.454{]} \\
      90 \% CI & {[} NA, 10.151 {]} & {[}-0.507, 15.387{]} & {[}-0.267, 8.551{]} & {[}-0.234, 20.891{]} & {[}-17.423, 392.691{]} \\
      Germany & {[}0.271, 1.742{]} & {[}0.0, 2.267{]} & {[}0.349, 1.432{]} & {[}0.23, 2.328{]} & {[}26.381, 38.407{]} \\
      90 \% CI & {[}-0.483, 13.529{]} & {[}NA, 10.073{]} & {[}-0.375, 8.812{]} & {[}-0.562, 21.558{]} & {[}-24.019, 250.719{]} \\
      Spain & {[}0.123, 1.381{]} & {[}0.099, 0.959{]} & {[}0.0, 0.939{]} & {[}0.094, 1.217{]} & {[}1.591, 21.227{]} \\
      90 \% CI & {[}-0.38, 20.141{]} & {[}-0.411, 11.789{]} & {[}NA, 4.325{]} & {[}-0.221, 16.117{]} & {[}-8.269, 231,428{]} \\
      UK & {[}0.21, 3.976{]} & {[}0.169, 5.019{]} & {[}0.158, 2.586{]} & {[}0.0, 2.396{]} & {[}6.683, 47.173{]} \\
      90 \% CI & {[}-0.561, 29.971{]} & {[}-0.461, 43.08{]} & {[}-0.34, 15.319{]} & {[}NA, 15.698{]} & {[}-11.701, 305.055{]} \\
      US & {[}0.865, 5.078{]} & {[}0.873, 5.204{]} & {[}0.974, 4.11{]} & {[}2.494, 8.527{]} & {[}0.0, 54.242{]} \\
      90 \% CI & {[}-4.451, 16.651{]} & {[}-6.341, 16.399{]} & {[}-5.012, 10.112{]} & {[}-20.314, 22.419{]} & {[}NA, 96.179{]}\\
      \hline
    \end{tabular}}
  \caption{Bound estimates of fixed costs varying by origin and
    destination (in million USD)}
  \label{tab:res3}
\end{table}

Table \ref{tab:res4} shows the bound estimates of fixed costs varying
by destination and time-period. I split the same in to two time
periods (2002-06) and (2006-09). The bound estimates do not seem to
reflect that the costs are increasing over time however, the
confidence interval widens once we move from the first time period to
the other time period. The 90\% confidence intervals widen as we have
smaller sample size.
\begin{table}[htbp!]
  \floatintro{The table shows the bound estimates of fixed costs
    varying by destination and time-period. I split the sample in to
    two time periods (2002-06) and (2006-09). 90\% confidence
    intervals are reported below the bounds. 10 million draws were
    used in the estimation of confidence intervals. All values are
    reported in mullions of USD.}
  \centering
  \resizebox{\textwidth}{!}{
    \begin{tabular}{lrrrrr}
      \hline
      & France & Germany & Spain & UK & US \\
      \hline
      2002 - 06 & {[}0.301, 2.697{]} & {[}0.189, 4.124{]} & {[}0.179, 2.328{]} & {[}0.257, 6.153{]} & {[}10.815, 53.877{]} \\
      90 \% CI & {[}-1.031, 11.219{]} & {[}-0.659, 20.316{]} & {[}-0.752, 9.575{]} & {[}-2.718, 31.637{]} & {[}-23.431, 178.514{]} \\
      2006 - 09 & {[}0.287, 3.344{]} & {[}0.238, 3.613{]} & {[}0.237,2.612{]} & {[}0.275, 4.679{]} & {[}11.631, 52.351{]} \\
      90 \% CI & {[}-1.631, 19.319{]} & {[}-1.519, 22.513{]} &
                                                               {[}-1.093,
                                                               13.571{]}
                                 & {[}-2.953, 27.212{]} & {[}-23.671,
                                                          206.211{]} \\ 
      \hline
    \end{tabular}}
  \caption{Bound estimates of fixed costs varying by destination and
    time periods (in million USD)}
  \label{tab:res4}
\end{table}
\section{Conclusion}
We estimate the fixed costs of exporting a movie abroad. The decision
to export is dependent on the expected revenues that a movie will
generate on exporting. We use predicted revenues to estimate the
bounds and confidence intervals.  We use the revealed preference
moment inequality approach to construct the estimates.\\
Our estimates shed some light on the fixed costs. First, we find that
the common fixed costs are in the range of 2.8-9.6 million
USD. Second, we find that the fixed costs are varying by destination
and are the highest when exporting to US. This makes sense as the
movies in other countries are in their languages which may not be able
to generate enough revenue to cover the fixed costs. However, the
costs are not that high in UK, even though it is an English speaking
country which might be reasonable considering the proximity of UK to
the European countries and the fact that it was a part of European
Union. Third, we also find that the fixed costs are varying by
destination and origin country. Finally, we split the sample into two
time periods and find that the fixed costs are not increasing over
time.
\newpage
\bibliography{papers}
\newpage
\begin{appendices}
  \section{Replication Code and Files} \label{appendix:a} The code for
  this exercise was written in \texttt{R} and \texttt{Julia}. The code
  is available on my Github
  repository\footnote{\url{https://www.github.com/dhananjayghei/io_estimation}}
  The code is present in the \texttt{src/} folder. The following files
  will run the code:
\begin{enumerate}
\item \texttt{final.R} - This is the \texttt{R} file that reads in the
  data, cleans it and generates tables for Section 3.2 (summary
  statistics)
\item \texttt{moment\_functions.jl} - This is a function file that has
  all the functions that are called in \texttt{moment.jl} for
  estimation.
\item \texttt{moment.jl} - This is the main \texttt{Julia} file that
  reads in the cleaned data, performs estimation and reports the bound
  estimates and the confidence intervals.
\end{enumerate}
\end{appendices}
\end{document}